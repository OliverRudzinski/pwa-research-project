\documentclass[
%===============================================================
%	DOCUMENT PREFERENCES
%===============================================================
	parskip		=	half,			% remove first-row indent in new paragraph
	headheight	= 	12pt,			% header height
	footheight 	= 	16pt,			% footer height
	headsepline,						% header separator line
	footsepline,						% footer separator line
	abstracton,						% Abstract headers
	headinclude	=	false,		
	footinclude	=	false,
	listof		=	totoc,			% List of ... in TOC
	toc			=	bibliography,	% Bibliography in TOC
	draft		=	false
]{scrreprt}

% Language and general symbol preferences
\usepackage[ngerman, english]{babel}
\usepackage[utf8]{inputenc}
\usepackage{soul}					% hyphening, etc.
\usepackage[super]{nth}				% superscript "n-th" (counting, etc.)
\usepackage{enumitem}
\usepackage{wasysym} 				% use additional symbols

% Page preferences
\usepackage[a4paper]{geometry}				
\geometry{
	a4paper,
	margin		=	2.5cm,
	foot 		= 	1.0cm
}

% Font preferences
\usepackage{couriers}
\KOMAoptions{fontsize=12pt}			% 12pt font size
\addtokomafont{disposition}			% Serif font for chapter headings
	{\rmfamily\bfseries}

\usepackage{setspace}				% 1.5x line spacing	
\onehalfspacing

% Header & Footer preferences
\usepackage{scrlayer-scrpage}		% Clear default settings
\pagestyle{scrheadings}
\clearpairofpagestyles

\ihead{Studienarbeit (T3101)}		% Header left	(Type)
\automark{chapter}					% Header right	(Chapter)
\ohead{\rightmark}
\ifoot{DHBW Stuttgart}				% Footer left	(School)
\cfoot{Goldschmidt, Rudzinski}				% Footer center	(Author)
\ofoot[\pagemark]{\pagemark}			% Footer right	(Page mark)

%===============================================================
%	ADDITIONAL PREFERENCES
%===============================================================

% Bibliography preferences
\usepackage[style=ieee]{biblatex}
\bibliography{back-matter/literature.bib}

% Management packages
\usepackage[titles]{tocloft}			% ToC management
\setlength{\cftbeforechapskip}{5pt}
\setcounter{secnumdepth}{3}			% Sub-subsection numbering
\usepackage{array}					% Table management
\usepackage{multirow}
\usepackage{tabularx} 				% for breaking table entries
\newcolumntype{C}{>{\centering\arraybackslash}X}
									% centered "X" column
\usepackage{tcolorbox}
\usepackage{colortbl}
\tcbuselibrary{skins}
\tcbset{tab2/.style={enhanced,fonttitle=\bfseries,fontupper=\normalsize\sffamily,
		colback=yellow!10!white,colframe=red!50!black,colbacktitle=Salmon!40!white,
		coltitle=black,center title}}

\usepackage{acronym}					% Acronym management
\usepackage{graphicx}				% Figure management
\usepackage{subfig}
\usepackage{hyperref}				% Referencing management
\hypersetup{colorlinks}
\usepackage{minted}					% Source code management
\setminted[json]{
	autogobble,
	baselinestretch=1,
	breaklines,
	frame=lines,
	fontsize=\footnotesize,
	framesep=3mm,
	linenos
}
\setminted[xml]{
	autogobble,
	baselinestretch=1,
	breaklines,
	frame=lines,
	fontsize=\footnotesize,
	framesep=3mm,
	linenos
}
\setminted[swift]{
	autogobble,
	baselinestretch=1,
	breaklines,
	frame=lines,
	fontsize=\footnotesize,
	framesep=3mm,
	linenos
}
\setminted[TypeScript]{
	autogobble,
	baselinestretch=1,
	breaklines,
	frame=lines,
	fontsize=\footnotesize,
	framesep=3mm,
	linenos
}
\setminted[ng2]{
	autogobble,
	baselinestretch=1,
	breaklines,
	frame=lines,
	fontsize=\footnotesize,
	framesep=3mm,
	linenos
}
\usemintedstyle{manni}	
\usepackage[T1]{fontenc}
\usepackage{inconsolata}			
	
% Misc
\setcounter{tocdepth}{3}				
\setcounter{lofdepth}{2}

% Kaviat Diagram
\usepackage{tkz-kiviat,numprint} 

%===============================================================
%	CUSTOM COMMANDS
%===============================================================

% Title page image handling
\newcommand*{\vcenteredhbox}[1]{
	\begingroup
	\setbox0=\hbox{#1}\parbox{\wd0}{\box0}
	\endgroup
}

% Remove page break on abstract
\newenvironment{absolutelynopagebreak}
{\par\nobreak\vfil\penalty0\vfilneg\vtop\bgroup}{\par\xdef\tpd{\the\prevdepth}\egroup\prevdepth=\tpd}

% Listing package for source code
\renewcommand{\listingscaption}{Quellcode-Ausschnitt}
\renewcommand{\listoflistingscaption}{Quellcodeverzeichnis}