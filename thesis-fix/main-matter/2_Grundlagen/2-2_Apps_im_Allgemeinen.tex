Anwendungen oder Programme, welche für Smartphones oder Tablets entwickelt wurden, werden gemeinhin als \textit{Apps} bezeichnet. Der Name stammt hierbei aus dem Englischen: \textit{App} ist die Kurzform von \textit{Application}. Ins Deutsche könnte man Application wörtlich mit \textit{Anwendung} übersetzen \cite{BegriffApp}. Es handelt sich dabei um \textit{High-Level}-Anwendungen mit umfangreicher Benutzeroberfläche. Im Gegensatz dazu stehen \textit{Low-Level}-Anwendungen, wie bspw. Gerätetreiber.

Mit der Annäherung von Desktop-Computern über Notebooks an Tablets, welche eine Vielzahl mobiler Geräte, wie Touch-Notebooks, 2-in-1 Notebooks oder Tablets mit Tastatur und vollwertigem Betriebssystem auf den Markt brachte, ist der Begriff \textit{App} nicht mehr scharf definiert.
Längst bezeichnet Microsoft Windows Anwendungen, die über den Microsoft Store heruntergeladen werden können, als \textit{Apps}. Es ist festzuhalten, dass im deutsches Sprachgebrauch \textit{mobile Apps} für Smartphones und Tablets meist abgekürzt als \textit{Apps} bezeichnet werden.

"`Je nach Betriebssystem gibt es verschiedene \textit{App-Stores}."', schreibt das Gabler Banklexikon \cite{BegriffAppGabler}. Sowohl unter Windows, als auch unter iOS und Android stellt der Betriebssystem-Hersteller Apps über einen Marktplatz zur Verfügung. Für dieses Konzept existieren mehrere Namen, im Umgangssprachlichen wird dieser Marktplatz aber meist als \textit{App Store} bezeichnet, obwohl dies eigentlich der Eigenname des Anwendungsmarktplatzes von Apple ist. 
Ein solcher Marktplatz unterstützt Nutzer dabei, gezielt Apps zu suchen und zu installieren. Der Marktplatzbetreiber wickelt Zahlungen zwischen Nutzer und App-Entwickler ab und bietet Nutzern die Möglichkeit, Apps zu bewerten. 
Die \acf{pwa} bricht mit diesem Konzept und lässt Nutzer eine Anwendung über den Browser installieren. 

Zu jeder App existiert eine Verknüpfung bzw. eine Startschaltfläche auf dem Startbildschirm des Betriebssystems. Sie bildet einen der wichtigsten Bestandteile, welcher mit dem Wort \textit{App} assoziiert wird: Ein einzigartiges, meist quadratisches Logo, dass die App per Fingertipp startet. Anschließend erwarten Nutzer einige Konventionen bezüglich der Nutzeroberfläche. Android Nutzer beispielsweise werden erwarten, dass der Hardwarebutton \textit{zurück} zur vorherigen Ansicht zurückleitet. Apple iOS Nutzer werden hingegen eine sichtbare Menüführung mit kleinen Pfeilen zum Wechseln der Ansichten erwarten. Alle Nutzer werden meist annehmen, dass eine App nur dann \textit{läuft} (also ausgeführt wird), solange er diese sieht und nicht geschlossen hat.

Zusammenfassend ist eine \textit{App} eine Anwendung für Mobilgeräte, welche aus einem Marktplatz heruntergeladen wird und sich an herstellerspezifische Designrichtlinien hält.