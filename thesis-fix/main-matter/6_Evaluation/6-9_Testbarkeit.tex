\textbf{Wertung native App}: \Circle \\
\textbf{Wertung \ac{pwa}}: $++$

\subsubsection{Bewertung native App}
Auch, wenn die Umsetzung von Tests in der Beispielanwendung aufgrund des eingeschränkten Rahmens dieser Arbeit nicht betrachtet wurde, bietet Xcode Schnittstellen für Unit- und \ac{ui}-Tests. Durch die einheitliche Plattform, welche die App unterstützt, kann davon ausgegangen werden, dass sich aus der Funktionsfähigkeit auf einem iOS-Gerät die Tüchtigkeit der App auf den übrigen Geräten ableiten lässt. Unter Android wäre diese Disziplin deutlich aufwändiger, da Android zwar ein einheitliches Betriebssystem darstellt, jedoch Eigenheiten aufgrund von Hardwareunterschieden des Gerätes oder Versionsunterschieden auftreten könnten. Ebenfalls negativ anzumerken ist das Problem, dass das reine Testen über den durch Xcode zur Verfügung gestellten iOS-Simulator nicht die gesamte Tragweite einer iOS-App darstellen kann (bspw. Kamera-Nutzung, etc.). Der Besitz eines physikalischen iOS-Gerätes ist erforderlich für das ganzheitliche Testen einer iOS-App.

Aufgrund der zuletzt genannten Nachteile ist die Testbarkeit neutral zu werten.

\subsubsection{Bewertung \ac{pwa}}
Ein Vorteil, welchen das JavaScript Ökosystem mit seinen Frameworks und Bibliotheken mit sich bringt, ist die Vielzahl existierender Testlösungen, bspw. für Unit-Testing. Ein gravierender Vorteil gegenüber nativer Apps ist, dass Entwickler die Anwendung im Desktopbrowser testen können und keinen Android- bzw. iOS-Emulator benötigen. Dahingehend ist das Testen der Anwendung deutlich einfacher, schneller und ressourcenschonender.

Die Testbarkeit wird deshalb mit gut bewertet.