\begin{tabbing}
	mmmmmmmmmmmmm				\= \kill
	\textbf{Wertung native App}: \> $+$ \\ 
	\textbf{Wertung \ac{pwa}}: \> $++$ 
\end{tabbing}

\subsubsection{Bewertung native App}
Wie auch bei der Installation administriert der zentrale Bezugspunkt (Apple App Store resp. Google Play Store) auch die Aktualisierung der auf dem Gerät installierten Apps. Je nach Nutzereinstellungen werden diese automatisch bezogen oder der Nutzer wird benachrichtigt, dass Updates verfügbar sind.

Wie bereits erwähnt werden alle Aktualisierungen, welche vom Entwickler vertrieben werden, vor Veröffentlichung erneut geprüft. Dies stellt einen Vor- wie einen Nachteil dar. Grundsätzlich kann aufgrund des Qualitätsmanagements der Bezugspunkte von einer sicheren Distribution ausgegangen werden. Andererseits können relevante Hotfixes dadurch auch verzögert werden. Sollte der Nutzer die automatischen Updates deaktiviert haben, so kann es sein, dass dieser sicherheitsrelevante Aktualisierungen nicht mitbekommt oder bewusst ignoriert, was nicht im Sinne der Entwickler ist.

Verglichen mit \acp{pwa} kann der Nutzer alle Updates genau nachverfolgen, da parallel zu diesen auch Informationen über die neuen Inhalte, Bug-Fixes, etc. veröffentlicht werden können. Sollte eine Sicherheitslücke einer neuen Version bekannt werden, so kann der Nutzer diese ignorieren, bis eine weitere Version veröffentlicht wird, welche diese behebt.

Alles in allem gewinnen gewisse Pro- bzw. Contra-Argumente an Bedeutung, je nachdem, welche Position gerade betrachtet wird. Zusammengefasst kann die Aktualisierbarkeit aber mit eher gut bewertet werden, da der Prozess an sich funktioniert und für den Nutzer in jeder Hinsicht transparent ist, jedoch unter gewissen Umständen ignoriert werden kann.

\subsubsection{Bewertung \ac{pwa}}
Der Browser bzw. der laufende Service-Worker verwaltet das Caching und die Aktualisierung der Anwendung. Der Nutzer wird über Aktualisierungen nicht informiert. Er muss ihnen nicht zustimmen und kann sie nicht vermeiden. Entwickler müssen nur die neue Version der Anwendung deployen, damit die Installationen auf den Nutzergeräten aktualisiert werden.

Da vom Nutzer keine Aktion erforderlich ist und die Aktualisierung für Entwickler sehr einfach ist, erhält dieses Kriterium die eine gute Wertung.