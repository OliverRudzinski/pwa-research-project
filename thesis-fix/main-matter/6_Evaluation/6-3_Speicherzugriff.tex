\textbf{Wertung native App}: $++$\\
\textbf{Wertung \ac{pwa}}: $+$

\subsubsection{Bewertung native App}
Die Beispielanwendung nutzt die Bibliothek \texttt{CoreData}, welche es erlaubt, zu speichernde Dateistrukturen in Form von Datenbank-Design anzulegen und in den Benutzungsrahmen der App hinzuzufügen. Innerhalb der Android-Entwicklung existieren verschiedene Formen app-bezogenen Speichers, welche für unterschiedliche Zwecke verwendet werden können \cite{AndroidStorage}. In beiden Fällen bedarf es \textit{Endpoints}, welche das Verhalten des Speicherzyklus definieren.

Durch eine Speicherarchitektur, welche an relationale Datenbanken erinnert, können auch komplexe Speicherstrukturen entstehen, bspw. über die Definition verschiedener Speicherareale, welche in der Beispielanwendung keinen Nutzen fanden. Anders als bei \acp{pwa} existieren keine Einschränkungen von Dateiformaten.

Das Kriterium kann somit uneingeschränkt als gut bewertet werden, da zumindest in der Beispielanwendung keinerlei Grenzen des Speicherzugriffes aufgedeckt werden konnten.


\subsubsection{Bewertung \ac{pwa}}
Die \ac{pwa} hat keinen Zugriff auf das Dateisystem. Daten werden über den Browser im Speicher abgelegt, bspw. im Key-Value-Store \texttt{local storage}. Konfigurationen und Daten können auf diese Weise einfach gespeichert werden.

Das Speichern von binären Daten, wie  bpsw. Bildern, gestaltet sich in der Praxis schwierig. Es existieren uneinheitliche, browserspezifische Lösungen. Höchstwahrscheinlich kommt der Entwickler aber nicht um das Speichern binärer Daten als kodierten Text, z.B. \texttt{Base64}.

Das Kriterium wird als eher gut bewertet, weil mit dem Browserspeicher ein Großteil der Anwendungsfälle für Datenspeicherung abgedeckt ist und diese sehr einfach von Entwicklern genutzt werden können.