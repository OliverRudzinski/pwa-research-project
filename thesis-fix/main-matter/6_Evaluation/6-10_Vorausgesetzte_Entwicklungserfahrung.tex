\begin{tabbing}
	mmmmmmmmmmmmm				\= \kill
	\textbf{Wertung native App}: \> $+$ \\
	\textbf{Wertung \ac{pwa}}: \> $-$\\
\end{tabbing}

\subsubsection{Bewertung native App}
Die Entwicklung einer iOS-App geschieht grundsätzlich innerhalb der Programmiersprache Swift. Diese greift jedoch auf Paradigmen des Vorgängers Objective-C zurück. Es werden zwar umfassende Kenntnisse \textit{nur} einer Programmiersprache benötigt, jedoch ist diese in ihrer Tragweite sehr umfangreich und folgt nicht immer den üblichen Entwicklungsmustern. Die Beispielanwendung lies sich ohne jegliche Drittanbieter-Bibliotheken umsetzen, weswegen die benötigten Dokumentationen der genutzten Funktionen überwiegend vom Betreiber Apple stammen, was zur Korrektheit und Aktualität dieser beiträgt. Positiv anzumerken ist ebenfalls die Tatsache, dass v.a. \ac{ui}-Konfigurationsschritte über den Interface Builder vereinfacht bzw. ersetzt werden können. Dies ist ebenfalls auf die Android-Entwicklung zu beziehen \cite{AndroidStudio}.

Da die Entwicklung in nativen Umgebungen durch verschiedene Lösungen vereinfacht wird und sich, verglichen mit \acp{pwa}, in \textit{einem} Umfeld aufhält, werden insgesamt weniger Voraussetzungen an den Entwickler gestellt. Auf der anderen Seite handelt es sich um sehr umfangreiche Programmierumgebungen, weswegen eine gewisse Erfahrung von Vorteil sein könnte. 

Insgesamt ist dieses Kriterium also mit eher gut zu bewerten.

\subsubsection{Bewertung \ac{pwa}}
Die \ac{pwa} setzt sich aus drei programmiersprachlichen Komponenten zusammen: JavaScript, \ac{html} und \ac{css}. Damit erfordert die Entwicklung sowohl Kenntnisse in prozeduraler Programmierung, als auch in der Implementierung passender \ac{html}- und \ac{css} Strukturen, welche letztendlich nur durch JavaScript modifiziert werden.

Außerdem ist zu erwähnen, dass komplexe JavaScript-Anwendungen ohne Frameworks und Bibliotheken in der Praxis selten zu finden sind. Die Nutzung von JavaScript ohne Angular, ReactJS, Vue.js, ö.Ä., ist mit nicht vertretbarem Aufwand verbunden. Da die Nutzung eines Frameworks quasi notwendig ist, aber es zwischen jenen deutliche Unterschiede gibt, zählt dies ganz klar zu den Wissensvoraussetzungen. Dazu kommen auch zwingend Kenntnisse der Linux-Kommandozeile für Node.js, \texttt{npm} und wahrscheinlich auch die eines \ac{cli}-Tools für das Deployment.

Die Wissenshürde ist deutlich erkennbar und für erfahrene Programmierer ohne Webkenntnisse dennoch vorhanden. Dieses Kriterium wird als negativ eingestuft.  