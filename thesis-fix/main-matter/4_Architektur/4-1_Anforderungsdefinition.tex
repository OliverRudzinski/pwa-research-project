Die Funktionalität der Anwendung wird zunächst über die Anforderungsdefinition näher beschrieben. Diese kann auf allgemeine Funktionsweisen der App (nicht-funktionale Anforderungen), sowie auf spezifische, technische Charakteristika abgegrenzter Bereiche der Anwendung (funktionale Anforderungen). Grundsätzlich gilt es, nicht-funktionale Anforderungen im Laufe der Anforderungsdefinition in meist mehrere funktionale Anforderungen zu überführen \cite{Garidis}, da diese qualifizierbarer sowie quantifizierbarer Natur sind und somit ebenfalls für eine bessere Vergleichbarkeit der zu entstehenden Anwendungen beitragen könnten.

\subsection{Nicht-Funktionale Anforderungen} \label{subsec:non-functional}
Die folgenden nicht-funktionalen Anforderungen beziehen sich auf Teile der Anwendung, welche jedoch abstrakter Natur sind, weswegen sie zunächst zu nicht-funktionalen Anforderungen gezählt werden müssen. Aufgrund der Formulierung werden diese Anforderungen auch als \textit{Nutzeranforderungen} (engl. \textit{User Requirements}) bezeichnet und stehen den spezifischeren, technisch versierteren \textit{Systemanforderungen} (engl. \textit{System Requirements}) gegenüber \cite{Garidis}.
\begin{description}
    \item[Anlegen, Auflisten, Bearbeiten und Löschen von Aufgaben] Die Anwendung ermöglicht es, Aufgaben hinzuzufügen. Die hinzugefügten Aufgaben werden aufgelistet. Bei Bedarf soll der Inhalt der Aufgabe nachträglich abgeändert werden können. Ebenfalls ist es möglich, die Aufgabe aus der Ansicht innerhalb der Anwendung zu entfernen.
    \item[Aufgaben bestehen nach Neustart der Anwendung bei] Wird die Applikation (gewollt und ungewollt) neu gestartet, bildet sie nach Neustart dieselben Aufgaben und Einstellungen wie zuvor ab.
    \item[Priorisierung der Aufgaben möglich] Bei Bedarf ist es möglich, einer bestimmten Aufgabe einen gesonderten Stellenwert zuzuweisen.
    \item[Benachrichtigungen über nicht-erledigte und überfällige Aufgaben] Der Nutzer wird unabhängig vom Status der Applikation oder des Smartphones (d.h. online/offline, geöffnet, im Hintergrund oder geschlossen bzw. gesperrt oder entsperrt) über nicht-erledigte und überfällige Aufgaben benachrichtigt.
\end{description}
Neben dieser Art der nicht-funktionalen Anforderungen koexisitieren jene, welche zwar ebenfalls abstrakt und allgemein gehalten sind, jedoch keinen Bedarf resp. keine Möglichkeit zur weiteren Spezifizierung an dieser Stelle des Prozesses aufweisen.
\begin{description}
    \item[Bereitstellung der Anwendung für mehrere Plattformen] Die Anwendung ist nicht nur auf einer Plattform verfügbar, sondern kann auf Geräten unterschiedlicher Betriebssysteme installiert und verwendet werden.
    \item[Aussehen und Verhalten sind deckungsgleich] Unabhängig davon, welche Plattform genutzt wird, ist die Interaktion zwischen dem Nutzer und der Anwendung annähernd identisch. Davon ausgenommen sind Aspekte, welche auf der entsprechenden Plattform nicht oder nur mit unverhältnismäßigem Aufwand erreicht werden können.
    \item[zeiteffizienter Entwicklungsprozess] Um auch einen entwicklungstechnischen Vergleich ziehen zu können, soll die Anwendung in einer dem Projekt angemessenen Zeit vollständig entwickelt werden können.
\end{description}
Die Problematik nicht-funktionaler Anforderungen im Bezug auf realistisch zu betrachtende Entwicklungsprojekte kann hier interpretiert werden. Vor allem bei eher unerfahrenen Entwicklern (zu welchen sich das Entwicklerteam dieses Projektes zu zählen erlaubt) sind bestimmte Tendenzen unklar. Dazu gehören bspw. das Bewusstsein über die Realisierbarkeit bestimmter Komponenten sowie die zeitliche Aufwandseinschätzung. Diese Störfaktoren werden im Laufe der Arbeit versucht, entkräftet zu werden und sind in die Evaluation der Forschungsfrage kritisch einzubeziehen.

\subsection{Funktionale Anforderungen}
Nichtsdestotrotz ist die Spezifizierung der in Abs. \ref{subsec:non-functional} eingangs definierten nicht-funktionalen Anforderungen noch ausstehend. Zur Unterstützung der Lesbarkeit werden diese in Reihenfolge der nicht-funktionalen Anforderungen abgehandelt.

\begin{description}
    \item[Bereitstellung klassischer \acs{crud}-Operationen] Sog. \textit{\ac{crud}}-Operationen greifen auf das Datenmodell der Anwendung zu. Diese erlauben die Manipulation der Daten auf Basis der gewünschten Operation. Diese Operationen sind unabhängig voneinander zu definieren und sinnvoll in den Verwendungsprozess der App einzubauen. Man spricht hier auch von sog. \ac{crud}-\textit{Endpoints}, welche vereinfacht als statische Funktionen beschrieben werden können.
    \item[Listendarstellung] Die Aufgaben sollen grundsätzlich in einer sortierten Liste dargestellt werden. Die Liste besteht aus individuellen Elementen, welche jeweils eine Aufgabe darstellen. Jene \ac{crud}-Operationen, welche speziell auf eine bestimmte Aufgabe angewandt werden sollen, finden ihre Aktivierung ebenfalls über ihre entsprechenden Elemente.
    \item[Bereitstellung eines Persistent Services] Bei Ausführen der zuvor definierten \ac{crud}-Operationen werden die Daten nicht nur in den flüchtigen Arbeitsspeicher des Smartphones geschrieben, sondern zugleich auch auf einen der Applikation zugewiesenen Festspeicher. Diese idealisierte Datenbank gleicht dem Datenmodell für die \ac{crud}-Operationen und wird somit bei jeder Ausführung dieser aktualisiert bzw. beansprucht.
    \item[Definition einer Hierarchie für Aufgaben] Eine hierarchische Struktur der Daten soll ermöglichen, bestimmte Aufgaben seitens der Anwendung anders zu behandeln als andere. Durch das Setzen eines sog. \textit{Flags} können die Aufgaben entsprechend der Hierarchiestruktur bestimmte Zustände übergeben bekommen, konkret eine hervorgehobene optische Darstellung innerhalb der \ac{ui} sowie die Präsentation an Anfang der Liste (Eingriff in die Sortierung der Aufgaben)
\end{description}


Streng genommen sind funktionale Anforderungen sehr granular zu definieren \cite{Garidis}. Da es sich hier jedoch um eine wissenschaftliche Arbeit handelt, und nicht um eine Entwicklerdokumentation, wird auf eine detaillierte Beschreibung verzichtet. Viel eher soll die nächste Sektion die genaue, weiterhin plattformunabhängige Umsetzung dieser Anforderungen erläutern, welche als Maßgabe für die spätere Entwicklung der Applikation dienen soll. \\\

Unabhängig von der Plattform wird zunächst ein allgemeines, während der individuellen Entwicklungsphase zu spezifizierendes Grundgerüst der Funktionalität definiert. Bei einer To-Do-Applikation besteht dieses grundsätzlich aus zwei Komponenten. Die idealisierte \textit{Datenbank} ermöglicht persistente Speicherung der angelegten To-Do-Einträge. Diese ist mit verschiedenen \ac{crud}-Funktionen direkt an das \ac{ui} angebunden und ermöglicht somit die Manipulation der Einträge.