\textbf{Wertung \ac{pwa}}: neutral \\
\textbf{Wertung native App}:  \\

\subsection{\ac{pwa}}
Die \ac{pwa} wird über den Browser installiert. Daher ist die Installation stark abhängig vom verwendeten Browsertyp. Teilweise ist die Installation der Anwendung bedingt durch den Browser überhaupt nicht möglich. Bezogen auf den Desktopbrowser Chrome dürfte es der Praxis häufig vorkommen, dass Nutzer die Bedienelemente zur Installation im Browser nicht als solche wahrnehmen (Siehe Abbildung \ref{fig:dialog_install_pwa_desktop}).Dies wird als negativ gewertet.

Die Installation ist für den Nutzer transparent. Er sieht nicht, dass beziehungsweise wie viele Daten bereits heruntergeladen wurden. Der ein oder andere Nutzer wird sich möglicherweise nicht bewusst sein, dass \texttt{zum Startbildschirm hinzufügen} eigentlich eine Installation ausführt.

Die Installation dauert je nach Komplexität der Webanwendung keine ganze bis wenige Sekunden.

Die Deinstallation auf Android Geräten erfolgt wie bei nativen Apps. Bei der Desktop \ac{pwa} ist diese unter Windows sogar deutlich einfacher, als bei Desktopanwendungen.

Insgesamt ist die Installation einer \ac{pwa} sehr einfach und mit nur einem Befehl des Nutzers ausführbar. Das gilt aber nur dann, wenn der Nutzer das Konzept der \ac{pwa} begriffen hat, was nach heutigem Stand wohl nicht der Fall sein dürfte. Die Installation selbst kommt ohne Lizenzvereinbarungen oder Installationspfade daher und ist außerdem bemerkenswert schnell.

Da einige störende Punkte für den Nutzer nicht optimal sind, die Installation insgesamt aber ein einfacher Prozess ist, erhält dieses Kriterium eine neutrale Wertung.
