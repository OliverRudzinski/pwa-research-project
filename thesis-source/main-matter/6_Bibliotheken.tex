\textbf{Wertung \ac{pwa}}: $++$\\
\textbf{Wertung native App}: $+$ \\

\subsubsection{\ac{pwa}}
Frameworks und Bibliotheken für JavaScript gibt es sprichwörtlich zu Tausenden. In der Praxis ist es ausgesprochen selten zu einem Problem keine existierende (Teil-)Lösung in Form eines npm Pakets zu finden.

Die Nutzung und Installation dieser Pakete ist mit dem npm Paketmanager einfach, automatisierbar und dürfte einen großen Teil zur Wahl der Programmiersprache JavaScript beitragen.

Dieses Kriterium ist eindeutig mit sehr gut zu werten.

\subsubsection{Bewertung native App}
Apple bietet für Swift eine Vielzahl von Bibliotheken an, eine Ergänzung bieten Drittanbieterbibliotheken, welche ebenfalls in die Entwicklungsumgebung eingebunden werden können. Gerade die direkte Unterstützung der eigenen Bibliotheken ermöglicht eine nahtlose Inklusion in den gesamten Entwicklungszyklus. Jedoch ist JavaScript wesentlich verbreiteter als Swift, weswegen das Volumen vorhandener Bibliotheken, welche die Entwicklung vereinfachen, deutlich größer ist. Die Android-Entwicklung ist jedoch ebenfalls sehr verbreitet, weswegen gerade dafür ebenfalls eine Vielzahl von Bibliotheken zur Verfügung steht. Da die bloße Anzahl der insgesamt vorhandenen Bibliotheken der einzige Punkt ist, in welchem die native Entwicklung der Webentwicklung (und somit der \ac{pwa}-Entwicklung) nachsteht, kann das Kriterium mit eher gut bewertet werden.