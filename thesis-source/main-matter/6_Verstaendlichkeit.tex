\textbf{Wertung \ac{pwa}}: neutral\\
\textbf{Wertung native App}:  \\



\subsection{\ac{pwa}}
Nutzer beziehen eine \ac{pwa} nicht über einen zentralen Markt, wie Apples App-Store, sondern über den Browser. Es ist davon auszugehen, dass Nutzer eine \ac{pwa} mehr aus Spontanität, als aufgrund gezielter Suche installieren. Es existiert, wie bereits erwähnt, kein Markt, auf welchem gezielt nach beispielsweise \texttt{Todo} gesucht werden kann. Es existieren nur einzelne Webanwendungen, die über eine Suchmaschine gefunden werden können. Vereinzelte Sammlungen von \ac{pwa}s existieren zwar, sind aber mit dem Bekanntheitsgrad des Apple oder Android App Markts nicht vergleichbar.

Es wird generell kein direkter Vor- oder Nachteil darin gesehen, dass ein Nutzer die \ac{pwa} von einer Webseite bezieht. Gegenüber einer nativen App, für welche eine Webseite auf einen externen App Markt verlinken muss ist dieser Prozess allerdings direkter und schneller.

Nicht zuletzt sollte erwähnt werden, dass für den Bezug einer \ac{pwa} kein Nutzerkonto benötigt wird.

Alles in Allem ist der Vertriebsweg für \ac{pwa}s noch sehr jung und gegenüber Apples AppStore tendenziell unausgereift. Allerdings besteht das Potenzial, dass Browserhersteller oder Suchmaschinen \ac{pwa}s mit in Suchergebnisse aufnehmen und so eine effektive Suche ermöglichen.

Dieses Kriterium befindet sich im neutralen Bereich. Da es jedoch die Möglichkeit der Installation von einer Webseite ohne einen Zwischenschritt gibt, wird dieses Kriterium als gut bewertet.