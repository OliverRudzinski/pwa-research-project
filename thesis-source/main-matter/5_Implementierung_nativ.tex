Die Entwicklung der nativen Mobilanwendung wird im Folgenden exemplarisch anhand einer \textit{iOS}-Applikation dargestellt. Diese werden ausschließlich auf Mobilgeräten vom U.S.-amerikanischen Hersteller \textit{Apple, Inc.} ausgeführt, funktionieren also auf diversen \textit{iPhone}-, \textit{iPod touch}- und in gewisser Hinsicht auch auf \textit{iPad}-Geräten. Letztere sind wegen der im September 2019 in Kraft getretenen Einführung des sog. \textit{iPadOS} nicht mehr auf reine iOS-Anwendungen ausgelegt, sondern nutzen diese lediglich in Form einer Vergrößerung der Kleingeräteanwendung.

Dieses Kapitel stellt im Folgenden den gesamten Entwicklungsprozess der To-Do-Anwendung dar, welche von der Wahl der Programmiersprache und -umgebung bis hin zur Ausführung auf verschiedenen kompatiblen Geräten reicht.

Der in Sektion 2.? angesprochene Standard heutiger iOS-Entwicklung wird auch in der Praxis dieser Untersuchung angewandt. Somit fällt die Wahl der Programmiersprache auf Apple Swift 5 unter der \acs{ide} Xcode 11. Im Hinblick auf die To-Do-Anwendung, welche einen persistenten Speicher benötigt, kann hier bereits die Integration der internen Bibliothek \textit{Core Data} hinzugefügt werden.

Grundsätzlich ist es sinnvoll, mit der Gestaltung der visuellen Ansicht zu beginnen, da die erstellten Komponenten im Nachhinein explizit im Quellcode referenziert werden können, um ihnen Funktionalität zu verleihen.

\subsection{Gestaltung des \acl{ui}}
Initial beinhaltet das \texttt{Main.storyboard} eine Ansicht, welche bereits mit dem automatisch generierten \texttt{ViewController} verknüpft ist. Diese muss nun mit den in Kapitel 3 beschriebenen Komponenten gefüllt werden. Die Listenansicht wird über einen \texttt{UITableView} realisiert. Dieser füllt einen großen Teil der Ansicht aus. Der untere Rand der Ansicht wird mit einem \texttt{UITextField} sowie mit einem \texttt{UIButton} versehen, welcher als \texttt{+}-Symbol konfiguriert wird. Somit ist die hierarchisch höchste Stufe der Ansicht fertiggestellt.

Bislang existiert noch keine Konfiguration der unteren Hierarchiestufe, welche aus den einzelnen Zellen des \texttt{UITableView} besteht. Da diese dem gleichen Aufbau folgen sollen, ist die Definition \textit{einer} \texttt{UITableViewCell} genügend, sodass diese dann für die gesamte Liste verwendet und repliziert werden kann. Diese Zelle beinhaltet erneut ein \texttt{UITextField}, sowie drei \texttt{UIButton}s, welche jeweils mit einem Kreuz-, Häkchen-, und Stern-Symbol konfiguriert werden.

Die Farbgebung der einzelnen Buttons wird nach Maßgabe der Farb-Definition aus Sektion 3.3.1 als \textit{Tint} der Buttons festgelegt.

Das finale \ac{ui} der To-Do-Anwendung, wie es im Interface Builder dargestellt wird, kann dem folgenden Screenshot entnommen werden.

\begin{figure}[h!]
	\centering
	\caption{\ac{ui} im Interface Builder von Xcode}
\end{figure} 

An dieser Stelle scheint es intuitiv, die Ansicht nun zu testen. Da bislang jedoch noch keinerlei Funktionalität implementiert ist, ist es nun wichtig, Zellen erstellen zu können. Für Testzwecke könnte eine provisorische Funktionsweise entwickelt werden, welche die Zelleninformationen zunächst im Arbeitsspeicher der Anwendung speichert. Da dies jedoch im Nachhinein zu großen Abänderungen des Quellcode-Aufbaus führen kann, werden zunächst der Persistent Service und die \ac{crud}-Funktionen definiert, welche dann direkt über den \texttt{ViewController} an das \ac{ui} angebunden werden.

\subsection{Entwicklung des Persistent Service}
Wie eingangs erwähnt geschieht die Konfiguration des Datenmodells des Persistent Service über die Konfigurationsdatei \texttt{to\_do.xcdatamodeld}. Dort kann eine neue Entität \texttt{ToDo} angelegt werden, welche die in Kapitel 3 beschriebenen Attribute (\texttt{id}, \texttt{text}, \texttt{done} und \texttt{priority}) in Form ihrer spezifizierten Datentypen beinhaltet. Da keine Beziehungen zu anderen Entitäten von Bedarf sind, kann diese Entität nun als sog. \texttt{NSManagedObject} exportiert werden. Dieses ermöglicht die Nutzung der Entität als Objekt im Quellcode und stellt notwendige Funktionen für die spätere \ac{crud}-Funktionalität bereit.

Die notwendigen Komponenten des Persistent Service liegen im \texttt{AppDelegate} bereit. In diesem wird davon ausgegangen, dass mehrere Instanzen eines solchen Services genutzt werden. Da es sich im Fall der To-Do-Anwendung jedoch um einen global gleich genutzten Speicher handelt, spricht nichts dagegen, die Komponenten in eine statische Klasse \texttt{Storage} auszulagern. In dieser Klasse werden nun auch Funktionen für die verschiedenen \ac{crud}-Operationen angelegt. Auf Basis der bereits vorhandenen Funktion \texttt{saveContext()} wird die Nicht-Flüchtigkeit dieser \ac{crud}-Operationen gewährleistet.

\subsubsection{\texttt{createToDo(...)}: Anlegen von To-Do-Elementen}
Der Aufbau der Funktion, welche für das Anlegen der To-Do-Elemente innerhalb des Persistent Service zuständig ist, wird hinsichtlich des zugrunde liegenden Quellcode-Ausschnitts beschrieben.


\begin{listing}[H]
	\inputminted{Swift}{sourcecode/ios_createToDo.swift}	
	\caption{Funktion zur Erstellung von To-Do-Elementen (Swift)}
\end{listing}

Es werden die Entitätsinformationen aus der Konfiguration der Datei \texttt{to\_do.xcdatamodeld} entnommen (Z. 2), sodass ein neues Objekt nach dieser Entität erstellt und dem Speicher zugeordnet werden kann (Z. 3). Daraufhin werden die initialen Werte der einzelnen Attribute gesetzt (Z. 6--9), wobei der beschreibende Text hier direkt aus dem Funktionsparameter entnommen wird. Zuletzt wird der Kontext (vereinfacht also der Speicher) aktualisiert und das neue Objekt wird in Form einer sog. \textit{Completion} zurückgegeben (Z. 1, Z. 13). Dieses kann in einer sog. \textit{Callback}-Funktion verwendet werden, welche nach erfolgreichem Durchlaufen der Ausgangsfunktion ausgeführt wird.

\subsubsection{\texttt{loadToDos(...)}: Laden von To-Do-Elementen aus dem Speicher}
Der Aufbau der Funktion, welche für das Laden der To-Do-Elemente aus dem Persistent Service zuständig ist, wird hinsichtlich des zugrunde liegenden Quellcode-Ausschnitts beschrieben.

\begin{listing}[H]
	\inputminted{Swift}{sourcecode/ios_loadToDos.swift}	
	\caption{Funktion zum Laden von To-Do-Elementen (Swift)}
\end{listing}

Zunächst wird der Datenabgriff vorbereitet, welcher auf die zuvor exportierte Klasse der \texttt{ToDo}-Entität zugreift (Z. 2). Auf Basis dieser Anfrage (Z. 5) wird ein Array vom Datentyp \texttt{ToDo} im gleicher Completion-Form wie zuvor zurückgegeben (Z. 6). Da dieser Vorgang \textit{Exceptions} schmeißen kann, muss dieser Vorgang entsprechend kontrolliert (d.\ h. mit \textit{Error Handling} versehen) ablaufen.

\subsubsection{Bearbeiten von Attributswerten von To-Do-Elementen}
Da das \ac{ui} mehrere Bedienelemete für das Bearbeiten eines Eintrags aufweist (d.\ h. das Ändern des Textes, der Priorität sowie des Status über Abschluss der Aufgabe), müssen entsprechend viele Funktionen für das Bearbeiten dieser Eigenschaften angelegt werden. In jeder der Funktionen wird das zu bearbeitende To-Do-Objekt mitgegeben und das zu ändernde Attribut wird mit dem neuen, ebenfalls mitgegebenen, Wert überschrieben. Das abschließende Ausführen von \texttt{saveContext()} sorgt für die Konsistenz des Objekts.

\subsubsection{\texttt{deleteToDo(...)}: Löschen von To-Do-Elementen}
Das mitgegebene To-Do-Objekt wird über die Kontext-Löschfunktion entfernt und der Stand des Kontext mit \texttt{saveContext()} gespeichert.

\subsection{Verknüpfung des \ac{ui} mit den \ac{crud}-Funktionen}