Die Datenbank muss auf eine Weise angelegt werden, dass ihre Daten (d.\ h. To-Do-Einträge) einem bestimmten Schema folgen. Konkret bedeutet dies folgende Attribute für die Entität \texttt{ToDo}:

% TODO Tabellarische Darstellung der Attribute
\begin{itemize}
	\item[\texttt{id}] (alpha-)numerische Zeichenfolge, welche einen Eintrag eindeutig erkennbar macht (String)
	\item[\texttt{text}] anzuzeigender Text, welcher den eigentlichen Eintrag darstellt und beschreibt (String)
	\item[\texttt{done}] Status über die Erledigung des entsprechenden Eintrages (Bool'scher Wert)
	\item[\texttt{priority}] Status über die Priorität des entsprechenden Eintrages (Bool'scher Wert)
\end{itemize}

% TODO Beispiel der idealisierten Datenbankeinträge
Unabhängig von der individuellen Architektur der jeweiligen Apps folgt dieses triviale Schema dem Konzept relationaler Datenbanken und könnte somit in einer einfachen Tabelle dargestellt werden.

Um auf die Datenbank zugreifen zu können, muss diese mit entsprechenden Funktionen ausgestattet werden. Neben dem bloßen Erstellen von Einträgen, müssen diese abgegriffen (engl. \textit{fetch}) sowie bearbeitet und gelöscht werden können. Die beiden letztgenanten Funktionen haben bei Ausführung nur Einfluss auf einen durch den Nutzer ausgewählten Eintrag. Somit müssen diese Funktionen das \textit{Objekt} des entsprechenden Eintrages übergeben bekommen. Weiterhin gliedert sich die Bearbeitung von Einträgen in drei Teilfunktionen auf, nämlich dem Ändern des \texttt{done}- oder \texttt{priority}-Attributes sowie dem Ändern des beschreibenden Textes des Eintrags.

Da die Ausführung sowie Umsetzung dieser Operationen mit der \ac{ui} Hand in Hand geht, wird die Definition dieser vorgezogen.
