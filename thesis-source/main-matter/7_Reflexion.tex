Ob sich die \ac{pwa} etablieren kann steht und fällt mit der zukünftigen Unterstützung durch Browserhersteller insbesondere Apples Safari aufgrund seiner hohen Marktverbreitung. Es ist davon auszugehen, dass die Mehrheit der Nutzer ihren Browser nicht nach Kriterien der JavaScript-Unterstützung wählt und sie für die Nutzung von \ac{pwa}s keinen neuen Browser herunterladen.

Es ist nach den Erkenntnissen dieser Arbeit jedoch fraglich, ob sich Apple grundsätzlich weiter mit \ac{pwa}s beschäftigt. Das Konzept bietet für Entwickler großes Potenzial, da verglichen mit nativer Implementierung häufig eine doppelte Programmierung für Web und Apps entfällt. Aber gerade deshalb steht die \ac{pwa} in Konkurrenz zu Apples AppStore. Mit der Entscheidung \ac{pwa}s zu unterstützen, gibt ein Hersteller die Kontrolle über die auf seinen Geräten installierten Apps ab. 
Möglicherweise würde man einen lukrativen Markt, den Verkauf von Software/Apps, in die Hände einzelner Unternehmen geben.

Für Entwickler jedenfalls wäre dies erfreulich, da im Hinblick auf die schnelle Ausbreitung von JavaScript im Webbereich quasi für jedes Problem bereits eine Lösung existiert. Nicht zuletzt kommuniziert fast jede native App sowieso mit dem Internet, um dynamisch Daten zu laden und zu speichern. Ein Webentwickler könnte mit einer \ac{pwa} eine App entwickeln, ohne Spezialist für eine Plattform zu sein. Unter Verwendung von Node.js könnten App, Website und Webservices alle in der gleichen Sprache, JavaScript, entwickelt werden.
