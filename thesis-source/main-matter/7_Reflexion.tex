Entgegen des üblichen Standards, die Ergebnisinterpretation vor der kritischen Würdigung der Arbeit durchzuführen, wird an dieser Stelle zunächst die Qualität der Forschungsfrage und ihre angewendete Umsetzung evaluiert, um die Ergebnisse des hervorgehenden Kapitels korrekt einzuordnen. Die Arbeit schließt mit einem Ausblick bezüglich der möglichen, folgenden Forschungsansätze.

\section{Kritische Würdigung}
Die Forschungsfrage wurde eingangs sehr abstrakt formuliert. Sie suggeriert eine ganzheitliche Betrachtung der mobilen App-Entwicklung. Auf der anderen Seite wurde für die Beispielimplementierung eine recht simple, funktionsbeschränke Architektur gewählt. Diese ist, im Nachhinein betrachtet, nur bedingt repräsentativ für die Gesamtheit an Möglichkeiten, die in der App-Entwicklung zum Tragen kommen und gewünscht sind.

Die Beispielimplementierung ist ebenfalls in ihrer Ausführung sehr eingeschränkt gewählt worden. So war die parallele Entwicklung einer \ac{pwa} in Kombination mit \textit{einer} nativen App für den Inhaltsumfang dieser Arbeit nachvollziehbar. Diesem fehlen jedoch weitere Alternativen, welche die Beantwortung der Forschungsfrage unterstützen könnten. Dazu gehört die Entwicklung einer architekturkonformen Android-App, um weniger Ableitungen und Abstraktionen bei der Evaluation ziehen zu müssen. Darüber hinaus existieren weitere Technologien, welche zwar zum Themengebiet dieser Arbeit gehören, jedoch ebenfalls aufgrund Umfangseinschränkung aus dem Betrachtungsrahmen entfallen sind. Dazu gehören Technologien wie die JavaScript-Umgebung \textit{React Native}, welche bereits in der Evaluation erwähnt wurde. Die gemeinsamen Vorteile von nativer sowie \ac{pwa}-Entwicklung werden dort bereits zusammengefügt, weswegen die Evaluation nur eine sehr einseitige Betrachtung und Beantwortung der Frage ermöglicht. Durch die Gesamtbetrachtung und Gewichtung der Kriterien wird dieses Problem zwar teilweise entkräftet, jedoch kann zum Ende dieser Arbeit keine definitive Beantwortung der Forschungsfrage in den Raum geworfen werden. Die entstehenden Kennzahlen müssten auf ihre Einzelteile zurückgeführt werden, sodass ebenfalls in Betracht gezogen werden sollte, ob bestimmte Kriterien mithilfe einer hybride Entwicklung (bspw. durch \textit{React Native}) noch weiter verstärkt werden könnten.

Durch die Definition der gewünschten Architektur der Beispielanwendung sowie die genaue Abgrenzung der Kriterien ist ein gewisser Vergleich dennoch möglich. Wägt man an dieser Stelle ausschließlich zwischen der nativen und der \ac{pwa}-Entwicklung ab, kann anhand der entstandenen Kennzahlen zumindest eine Handlungsempfehlung ausgesprochen werden. Ein höherer Wert für die \ac{pwa} würde die verstärkte Unterstützung dieser Plattform seitens Websitebetreibern suggerieren, während ein höherer Wert für die native Entwicklung den Status quo bevorzugt und eher auf die Entwicklung eigenständiger Apps setzen würde.

\section{Ergebnisinterpretation und -einordnung}
Rückblickend auf die Ergebniskennzahlen des Evaluationskapitels lassen sich unter Betrachtung der kritischen Würdigung der Forschungsmethode folgende Erkenntnisse schließen.

Die Entwicklung der \ac{pwa} befindet sich anhand ihrer Kennzahl recht genau zwischen den Werten \textit{neutral} und \textit{eher gut}. Die Entwicklung der nativen App, welche zu Teilen auch die Android-Entwicklung mit einbezieht, bewegt sich in einem \textit{eher guten} Rahmen. Die Nähe dieser Ergebnisse entkräftet deren Aussagekraft an dieser Stelle zusätzlich. Somit ist formal zwar eine leichte Tendenz zur nativen Entwicklung zu sehen, jedoch fehlen Erkenntnisse aus verwandten Bereichen dieses Themengebietes.

Es kann ebenfalls argumentiert werden, dass nicht genügend Faktoren für die verstärkte Entwicklung von \acp{pwa} sprechen. Auf Interpretationsebene gesprochen kann also gesagt werden, dass der Status quo ohne größere Bedenken bezogen auf den Markt weitergeführt werden kann. Andererseits spricht aber auch kein Argument gegen das zusätzliche Anbieten von \acp{pwa} seitens der Homepagebetreiber, sofern diese entsprechende Kapazitäten für deren Entwicklung aufweisen können.


Zusammengefasst kann die Forschungsfrage, ob native Apps durch \acp{pwa} langfristig ersetzt werden können, schwach mit nein beantwortet werden, jedoch gehen native Apps nicht als klarer Sieger hervor. Viel eher kann suggeriert werden, \acp{pwa} \textit{ergänzend} zu nativen Apps anzubieten, um das eigene Spektrum zu erweitern. Um diese Aussage jedoch fundierter zu belegen, bedarf es weiterer Forschung in diese Richtung, welche abschließend im Ausblick diskutiert wird.

\section{Ausblick}
Native Apps können zunächst \textit{nicht vollständig} durch \acp{pwa} ersetzt werden. Wie im ersten Abschnitt dieses Kapitels erwähnt, fehlen jedoch Forschungsansätze in verschiedenen Bereichen, um die abstrakt formulierte Fragestellung genauer und ambivalenzfreier beantworten zu können. Zum Einen müssten verschiedene Komplikationsstufen von Beispielanwendungen definiert und untereinander verglichen werden. Darüber hinaus müssen weitere Entwicklungsmethoden, vor allem die hybride sowie die Android-Entwicklung, in starken Betracht gezogen werden, da die Beantwortung sonst nur einseitig erfolgt.

Mit weiteren Ambitionen in die genannten Richtungen könnte die Forschungsfrage konkretisiert werden, um im Nachhinein Bereiche nennen zu können, in welchen \acp{pwa} potentiell eher Anwendung finden könnten als native Lösungen. Maßgebend dafür wäre ebenfalls die Frage, ob genannte Kompatibilitätsprobleme von \acp{pwa} gelöst werden könnten, da sie sonst, v.a. unter iOS, keine Konkurrenz darstellen.