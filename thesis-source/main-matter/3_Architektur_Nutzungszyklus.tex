Der Nutzer öffnet die Anwendung. Beim Laden der Ansicht wird eine Datenbank-Funktion ausgeführt, um bereits existierende Einträge zu laden und ihre entsprechenden Beschreibungen samt weiteren Attributen in die einzelnen Zellen der Listenansicht zu laden. 

Der Nutzer möchte einen neuen To-Do-Eintrag erstellen. Dafür wird ein beschreibender Text in das Texteingabefeld am unteren Rand der Ansicht geschrieben. Um einen neuen Eintrag zu erstellen, wird ein beschreibender Text in das Texteingabefeld am unteren Rand der Ansicht geschrieben. Mit der Bestätigung über den \texttt{+}-Button wird ein neuer Eintrag in die Datenbank aufgefordert, wobei das \texttt{text}-Attribut mit dem zuvor gewählten Text aus dem Eingabefeld gefüllt wird. Die Zeichenfolge der \texttt{id} wird automatisch und zufallsbasiert generiert, \texttt{done} sowie \texttt{priority} standardmäßig auf \texttt{false} gesetzt. Der neue Eintrag wird unter den bereits vorhandenen Einträgen angefügt. Stern- sowie Häkchensymbol werden lediglich über ihre Kontur kenntlich gemacht.

Der Nutzer möchte den Test eines Eintrages ändern. Durch direktes Tippen auf den dargestellten Text in einer Zelle wird dies ermöglicht. Es erscheint ein Cursor, welcher ebenfalls Tastatur- und Berührgesten innerhalb dieses Feldes ermöglicht. Nach bestätigter Änderung über das Verlassen des Textfeldes (d.\ h. dem Tippen auf eine andere Stelle innerhalb der Ansicht) wird erneut eine Datenbank-Funktion ausgeführt. Diese bekommt das \texttt{ToDo}-Objekt übergeben und ersetzt den bestehenden Inhalt des \texttt{text}-Attributes mit dem nun geänderten.

Der Nutzer möchte den zuvor erstellten Eintrag priorisieren. Tippt dieser auf das Stern-Symbol, wird dieses mit der zuvor definierten Farbe gefüllt. Gleichzeitig bewegt sich das priorisierte Element an das Ende der Teilliste mit priorisierten Einträgen. Eine Datenbank-Funktion wird aufgerufen, welche den \texttt{priority}-Wert des übergebenen Objekts von \texttt{false} auf \texttt{true} setzt.

Der Nutzer möchte den zuvor priorisierten Eintrag als abgeschlossen markieren. Tippt dieser auf das Häkchen-Symbol wird dieses mit der zuvor definierten Farbe gefüllt. Eine Datenbank-Funktion wird aufgerufen, welche den \texttt{done}-Wert des übergebenen Objekts von \texttt{false} auf \texttt{true} setzt.

Der Nutzer möchte den zuvor als abgeschlossenen Eintrag aus der Liste der priorisierten Einträge entfernen. Tippt dieser auf das Stern-Symbol, wird dessen Füllung entfernt. Gleichzeitig bewegt sich das Element an den Anfang der Teilliste mit nicht-priorisierten Einträgen. Die zuvor ausgeführte Datenbank-Funktion wird erneut aufgerufen, welche den \texttt{priority}-Wert des übergebenen Objekts von \texttt{true} nun wieder auf \texttt{false} setzt.

Der Nutzer möchte den Eintrag abschließend entfernen. Tipps dieser auf das Kreuz-Symbol, wird der Eintrag aus der Liste entfernt. Die Elemente unterhalb des gelöschten Eintrags verschieben sich um jeweils eine Position nach oben. Eine Datenbank-Funktion wird aufgerufen, welche das übergebene Objekt des Eintrages aus der Datenbank entfernt.
