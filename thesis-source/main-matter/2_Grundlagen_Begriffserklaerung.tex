In der Praxis sind viele, in dieser Arbeit häufig verwendete, Begriffe unterschiedlich belegt. Aus diesem Grund soll im Folgenden eindeutig klargestellt werden, was mit den verwendeten Begriffen tatsächlich gemeint ist.

\begin{description}
	\item [App (plural: Apps)]
		In dieser Arbeit werden Programme die speziell für Smartphones entwickelt wurden als Apps bezeichnet. Implizit wird hiermit auch die Plattform auf Android oder iOS eingrenzt. Der Begriff App zeichnet sich in dieser Arbeit dadurch aus, dass die damit gemeinte Anwendung für den Nutzer sehr einfach zu installieren ist. In der Regel beziehen Nutzer Apps aus einem Shop des Herstellers beispielsweise Apples App-Store oder Google Play und starten diese über ein Icon auf dem Startbildschirm des Betriebssystems.
		
	\item [Webseite, Webanwendung und Web App]
		Der Begriff Webseite wird in dieser Arbeit für HTML-basierte Inhalte verwendet, die der Nutzer über den Browser abruft.
		
		Die Webanwendung unterscheidet sich dahingehend, dass sie dynamisch auf den Nutzer reagiert, seine Eingaben auswertet und gegebenenfalls den angezeigten Inhalt ändert oder nachlädt. Speziell werden JavaScript basierte Anwendungen in dieser Arbeit als Webanwendung oder Web App bezeichnet. Web App und Webanwendung werden synonym verwendet.
		
		Eine Webseite kann, aber muss keine Webanwendung oder Web App sein.
	
	\item [Progressive Web App]
		Eine Progressive Web App ist eine Webseite und speziell eine Webanwendung oder Web App, welche dynamisch auf den Nutzer reagiert.
		In dieser Arbeit werden Webanwendungen, welche lokal auf einem Gerät installiert werden können als \acf{pwa} bezeichnet. Die PWA erfüllt die Kriterien des nachfolgenden Unterkapitels.
		
		Im Unterschied zur nativen App kann dieselbe \ac{pwa} sowohl auf Smartphones, als auch auf eine Desktopgerät (Notebook, Desktop Computer etc.) installiert werden.
		
	\item [Desktop \ac{pwa}]
		Mit Desktop \ac{pwa} ist hier explizit eine Progressive Web App gemeint, welche auf einem Desktopgerät installiert wird.
			
	\item [native App]
		Diese Arbeit beschäftigt sich mit einer modernen Methode Mobilanwendungen zu programmieren: der \ac{pwa}. Im Unterschied dazu ist eine native App in Java oder Swift geschrieben und ist damit stark plattformabhängig. Nativ implementiere Apps sind entweder für iOS oder Android entwickelt worden, nicht aber für mehrere Plattformen.
		
	\item [Container]
		Da die Entwicklung von Apps stark am Frontend orientiert ist, wird häufig der Begriff Container verwendet. Damit ist explizit \textit{kein Container im Sinne von Virtualisierung}, wie beispielsweise ein Docker-Container, gemeint. Der Begriff wird im HTML-Kontext verwendet und bezeichnet in dieser Arbeit ein Element, dass andere Elemente beinhaltet.
	
\end{description}
