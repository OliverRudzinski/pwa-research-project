\begin{absolutelynopagebreak}
	\begin{abstract}
    	Apps, also Anwendungen für Smartphones, werden über einen Marktplatz des Betriebssystem-Herstellers vertrieben. Die \acf{pwa} bricht mit diesem Konzept und lässt Nutzer eine Webanwendung über den Browser auf dem Gerät installieren. Damit verspricht die \ac{pwa} einige Vorteile Gegenüber nativen Apps, wie Plattformunabhängigkeit und Verwendung von etablierten Webtechnologien.
    	
    	Diese Arbeit vergleicht die Entwicklung nativer Apps mit derer \ac{pwa}s und evaluiert die Technologien anhand mehrerer Kriterien. Dabei wird auf die technischen Grundlagen eingegangen und exemplarisch eine Todo-Anwendung für iOS und als \ac{pwa} implementiert, welche anschließenden Evaluationsprozess stützt.
	\end{abstract}
	
	\selectlanguage{english}
	
	\begin{abstract}
		Mobile apps are distributed via a marketplace managed by the OS vendor. The \acf{pwa} breaks with this concept and allows users to install web applications on the device using the browser.
		The \ac{pwa} promises several advantages over native mobile apps, such as platform independence
		and use of established web technologies.
		This paper compares the development of native apps compared to PWAs and evaluates the
		technologies based on several criteria. 
		The technical basics are discussed and an exemplary todo-application for iOS and as \ac{pwa} is implemented,
		which supports the subsequent evaluation process.
	\end{abstract}
\end{absolutelynopagebreak}

